\documentclass[UTF8]{report}   
\begin{document}

\title{%
  Solving the Traveling Salesman Problem with Genetic Algorithms 
}

\author{Daniel Kearney}

\maketitle

\tableofcontents
\pagebreak

\section{Overview}

\subsection{History of the Genetic Algorithm}

\subsection{The Traveling Salesman Problem}

\section{Genetic Algorithm Function}

The genetic algorithm works by mimicking the mechanism of chromosomes and genes in evolutionary biology. The algorithm works by creating a random `generation' of solutions to seed the iterative process. Each individual solution -- called a `chromosome' -- has a random solution to the problem at hand, at first. 

The iterative step of the process works by using a method to pick the fittest members of each generation and selecting them as parents for the next generation. Pairs of parents are `crossed over', where the parent solutions are co-mingled, and finally, a bit of randomness is added by `mutating' some of the solutions with random adjustments. The process then repeats, for a certain number of generations or until some fitness level is reached. 

\section{Genetic Algorithm Implementation in Python}

\subsection{Selection}


\subsection{Crossing Over}

\subsection{Mutation}

\section{Complexity Analysis}

\section{Results}

\subsection{Small Dataset}

\subsection{Large Dataset}

\subsection{Large Real-world Dataset}

\end{document}